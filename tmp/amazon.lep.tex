\section{Question 1: Duplicate}

\subsection{Analysis}

Suppose $rows$ is the number of rows and $dist$ is the maximum distance. A
primitive method is to scan the neighbors within distance $dist$ of each
element and compare. A minor optimization is to only check the right half of
neighbors, which eliminates redundant checks. The space complexity would be
$O(rows^2)$. When $dist < rows$ the time complexity is $O(rows^2 * dist^2)$,
and when $dist \geq m$ the time complexity is $O(rows^4)$.

The time complexity can be reduced by using a HashMap, where each key is an
integer that has been encountered, and each value is a linked list storing all
corresponding coordinates in the past $d$ rows. Additional space is required
for up to $rows^2$ keys and up to $rows^2$ coordinate nodes. When $0 < dist <
rows$, each node needs to be compared with at most roughly $2 * rows / dist$
nodes, thus the time complexity is $O(rows^3 / dist)$. When $dist \geq rows$,
each node needs to be compared with at most 4 nodes, so the time complexity is
$O(rows^2)$.

I thought about using HashMap while doing it online, but was afraid that there
would not be enough time to work out the details of the linked list. However it
turned out that dealing with the boundaries while applying a more primitive
method actually consumed quite a lot of time, and a few mistakes prevented it
from working properly.

\subsection{Java Code}

The structure of class $Duplicate$:
\begin{lstlisting}[title={<src/duplicate/Duplicate.java 1>}, label=Listing1]
package duplicate;

import java.util.HashMap;
import java.util.Scanner;

public class Duplicate {

    <<Coordinate Node Class 4>>

    <<HashMap Insertion 6>>

    <<Duplicate Check 5>>

    <<Matrix Output 3>>

    <<Duplicate Main 2>>

}
\end{lstlisting}\begin{footnotesize} \textsc{Included Blocks}: \hyperref[Listing4]{4 on page} \pageref{Listing4}, \hyperref[Listing6]{6 on page} \pageref{Listing6}, \hyperref[Listing5]{5 on page} \pageref{Listing5}, \hyperref[Listing3]{3 on page} \pageref{Listing3}, \hyperref[Listing2]{2 on page} \pageref{Listing2}\end{footnotesize}\vskip 5mm\noindent
The main function:
\begin{lstlisting}[title={<Duplicate Main 2>}, label=Listing2]
    public static void main(String[] args) {
        // reads in rows, matrix, and dist
        Scanner sc = new Scanner(System.in);
        int rows = sc.nextInt(); // number of rows
        int[][] matrix = new int[rows][];
        for (int i = 0; i < rows; i++) {
            matrix[i] = new int[rows];
            for (int j = 0; j < rows; j++) {
                matrix[i][j] = sc.nextInt();
            }
        }
        int dist = sc.nextInt(); // maximum distantce
        sc.close();
        System.out.println(hasDuplicateInDist(matrix, dist) ? "YES" : "NO");
    }
\end{lstlisting}\begin{footnotesize} \textsc{Used in}: \hyperref[Listing1]{s\-r\-c\-/\-d\-u\-p\-l\-i\-c\-a\-t\-e\-/\-D\-u\-p\-l\-i\-c\-a\-t\-e\-.\-j\-a\-v\-a on page} \pageref{Listing1}  \end{footnotesize}\vskip 5mm\noindent
Matrix output:
\begin{lstlisting}[title={<Matrix Output 3>}, label=Listing3]
    /**
     * Outputs an integer matrix.
     * @param matrix the integer matrix
     */
    public static void outputMatrix(int[][] matrix) {
        for (int i = 0; i < matrix.length; i++) {
            for (int j = 0; j < matrix[i].length; j++) {
                System.out.print(matrix[i][j] + (j < matrix[i].length - 1 ? " " : ""));                
            }
            System.out.println();
        }
    }
\end{lstlisting}\begin{footnotesize} \textsc{Used in}: \hyperref[Listing1]{s\-r\-c\-/\-d\-u\-p\-l\-i\-c\-a\-t\-e\-/\-D\-u\-p\-l\-i\-c\-a\-t\-e\-.\-j\-a\-v\-a on page} \pageref{Listing1}  \end{footnotesize}\vskip 5mm\noindent
\begin{lstlisting}[title={<Coordinate Node Class 4>}, label=Listing4]
    /**
     * Coordinate node class.
     */
    public static class cNode {
        int row, column;
        cNode next;

        /**
         * Default constructor.
         */
        public cNode() {
            this.row = 0;
            this.column = 0;
            this.next = null;
        }

        /**
         * Constructor with coordinates.
         */
        public cNode(int row, int column) {
            this.row = row;
            this.column = column;
            this.next = null;
        }

        /**
         * Calculates distance to another node
         */
        public int dist(cNode node) {
            return Math.abs(this.row - node.row) + Math.abs(this.column - node.column);
        }

    }
\end{lstlisting}\begin{footnotesize} \textsc{Used in}: \hyperref[Listing1]{s\-r\-c\-/\-d\-u\-p\-l\-i\-c\-a\-t\-e\-/\-D\-u\-p\-l\-i\-c\-a\-t\-e\-.\-j\-a\-v\-a on page} \pageref{Listing1}  \end{footnotesize}\vskip 5mm\noindent
Here is the method that inserts all elements and checks for duplicates.
\begin{lstlisting}[title={<Duplicate Check 5>}, label=Listing5]
    /**
     * Checks if there are two elements in a matrix
     * with distance less than a given integer.
     */
    public static boolean hasDuplicateInDist(int[][] matrix, int dist) {
        int rows = matrix.length;
        HashMap<Integer, cNode> map = new HashMap<Integer, cNode>();
        // inserts and checks all elements
        for (int i = 0; i < rows; i++) {
            for (int j = 0; j < rows; j++) {
                if (insertAndCheck(map, matrix[i][j], new cNode(i, j), i, dist)) {
                    return true;
                }
            }
        }
        return false;
    }
\end{lstlisting}\begin{footnotesize} \textsc{Used in}: \hyperref[Listing1]{s\-r\-c\-/\-d\-u\-p\-l\-i\-c\-a\-t\-e\-/\-D\-u\-p\-l\-i\-c\-a\-t\-e\-.\-j\-a\-v\-a on page} \pageref{Listing1}  \end{footnotesize}\vskip 5mm\noindent
This method inserts a coordinate into a HashMap and checks for duplicate.
\begin{lstlisting}[title={<HashMap Insertion 6>}, label=Listing6]
    /**
     * Inserts a coordinate node to a HashMap, 
     * and checks its distance with previous coordinates.
     * @param  map        the HashMap
     * @param  key        the integer key
     * @param  currentRow the index of current row
     * @param  dist       maximum distance
     * @return <tt>true</tt> only when it is square.
     */
    public static boolean insertAndCheck(HashMap<Integer, cNode> map, int key, cNode newNode, int currentRow, int dist) {
        if (map.containsKey(key)) { // already encountered
            cNode head = map.get(key);
            // remove nodes not in the past dist rows
            while (head != null && head.row < currentRow - dist) {
                head = head.next;
            }
            if (head == null) { // linked list empty
                head = newNode;
            } else { // linked list not empty
                cNode cursor = head;
                // check the distances with nodes in the linked list
                while (cursor.next != null) {
                    if (newNode.dist(cursor) <= dist) {
                        return true;
                    }
                    cursor = cursor.next;
                }
                // if found
                if (newNode.dist(cursor) <= dist) {
                    return true;
                }
                // insert to the tail
                cursor.next = newNode;
            }
            // update map
            map.put(key, head);
        } else { // not encounted yet
            map.put(key, newNode);
        }
        // not found
        return false;
    }
\end{lstlisting}\begin{footnotesize} \textsc{Used in}: \hyperref[Listing1]{s\-r\-c\-/\-d\-u\-p\-l\-i\-c\-a\-t\-e\-/\-D\-u\-p\-l\-i\-c\-a\-t\-e\-.\-j\-a\-v\-a on page} \pageref{Listing1}  \end{footnotesize}\vskip 5mm\noindent
\subsection{Test Cases}
\begin{lstlisting}[title={<duplicate0.txt 7>}, label=Listing7]
0
0
\end{lstlisting}
\begin{lstlisting}[title={<duplicate1.txt 8>}, label=Listing8]
1
1
1
\end{lstlisting}
\begin{lstlisting}[title={<duplicate2-0.txt 9>}, label=Listing9]
2
1 2
3 4
2
\end{lstlisting}
\begin{lstlisting}[title={<duplicate2-1.txt 10>}, label=Listing10]
2
1 2
3 1
2
\end{lstlisting}
\begin{lstlisting}[title={<duplicate2-2.txt 11>}, label=Listing11]
2
1 2
2 4
2
\end{lstlisting}\begin{lstlisting}[title={<duplicate2-3.txt 12>}, label=Listing12]
2
1 2
3 1
1
\end{lstlisting}
\begin{lstlisting}[title={<duplicate2-4.txt 13>}, label=Listing13]
2
1 2
2 4
1
\end{lstlisting}
\begin{lstlisting}[title={<duplicate5-0.txt 14>}, label=Listing14]
5
1 2 3 4 5
6 7 8 9 10
11 12 13 14 15
16 17 18 19 20
21 22 23 24 25
5
\end{lstlisting}
\begin{lstlisting}[title={<duplicate5-1.txt 15>}, label=Listing15]
5
1 2 3 4 5
6 7 8 9 10
11 12 13 14 15
16 17 18 19 20
21 22 23 24 1
8
\end{lstlisting}
\begin{lstlisting}[title={<duplicate5-2.txt 16>}, label=Listing16]
5
1 2 3 4 5
6 7 8 9 10
11 12 13 14 15
16 17 18 19 20
5 22 23 24 25
8
\end{lstlisting}
\begin{lstlisting}[title={<duplicate5-3.txt 17>}, label=Listing17]
5
1 2 3 4 5
6 7 8 9 10
11 12 13 14 15
16 17 18 19 20
21 22 23 24 1
4
\end{lstlisting}
\begin{lstlisting}[title={<duplicate5-4.txt 18>}, label=Listing18]
5
1 2 3 4 5
6 7 8 9 10
11 12 13 14 15
16 17 18 19 20
5 22 23 24 25
4
\end{lstlisting}
\begin{lstlisting}[title={<duplicate5-5.txt 19>}, label=Listing19]
5
1 2 3 4 5
6 7 8 9 10
11 12 13 14 15
16 17 18 19 20
1 22 23 24 25
4
\end{lstlisting}
\begin{lstlisting}[title={<duplicate5-6.txt 20>}, label=Listing20]
5
1 2 3 4 1
6 7 8 9 10
11 12 13 14 15
16 17 18 19 20
21 22 23 24 25
4
\end{lstlisting}
\begin{lstlisting}[title={<duplicate5-7.txt 21>}, label=Listing21]
5
1 2 3 4 25
6 7 8 9 10
11 12 13 14 15
16 17 18 19 20
21 22 23 24 25
4
\end{lstlisting}
\begin{lstlisting}[title={<duplicate5-8.txt 22>}, label=Listing22]
5
1 2 3 4 5
6 7 8 9 10
11 12 13 14 15
16 17 18 19 20
25 22 23 24 25
4
\end{lstlisting}
\begin{lstlisting}[title={<duplicate5-9.txt 23>}, label=Listing23]
5
1 2 3 4 5
6 7 8 9 2
11 12 13 14 15
2 17 18 19 20
21 22 23 2 25
4
\end{lstlisting}
\begin{lstlisting}[title={<duplicate5-10.txt 24>}, label=Listing24]
5
1 2 3 4 5
6 7 8 9 2
11 12 13 14 15
2 17 18 19 20
21 22 23 2 25
3
\end{lstlisting}
\begin{lstlisting}[title={<duplicate5-11.txt 25>}, label=Listing25]
5
1 2 3 4 1
6 7 8 9 2
11 12 1 14 15
2 17 18 19 20
1 22 23 2 1
4
\end{lstlisting}
\begin{lstlisting}[title={<duplicate5-12.txt 26>}, label=Listing26]
5
1 2 3 4 1
6 7 8 9 2
11 12 1 14 15
2 17 18 19 20
1 22 23 2 1
3
\end{lstlisting}\begin{lstlisting}[title={<duplicate5-13.txt 27>}, label=Listing27]
5
1 1 3 4 5
6 7 8 9 10
11 12 13 14 15
16 17 18 19 20
21 22 23 24 25
3
\end{lstlisting}\begin{lstlisting}[title={<duplicate5-14.txt 28>}, label=Listing28]
5
1 2 3 4 5
6 7 8 9 10
11 12 13 14 15
16 17 18 19 20
21 22 23 25 25
3
\end{lstlisting}
\section{Question 2: Rotate Array}

\subsection{Analysis}

Suppose $rows$ is the number of rows. The matrix squareness check can be done
when reading each row, by comparing the number of integers in the current row
with $rows$.

For a square matrix, a primitive method would read it in, apply the rotation,
and output the result. Its time complexity and space complexity are both
$O(rows^2)$. Each element is read once (before moving) and written twice
(initializing and after moving, once each). If we want to have a general
matrix-to-matrix operation they are optimal. This was what I used when doing it
online, but the time was not enough to take care of the details.

However, if we consider to make the solution specifically as a stream-to-stream
operation, the time complexity can be easily reduced by calculating the new
coordinates of each element while reading from the input stream, and store the
element directly to the new location. The operation on each element is reduced
to just writing once. The time complexity and space complexity remain
$O(rows^2)$.

The structure of a square matrix with odd $rows$ is:\\
\begin{figure}[hbtp]
\begin{verbatim}
1 1 1 1 2
4 1 1 2 2
4 4 5 2 2
4 4 3 3 2
4 3 3 3 3
\end{verbatim}
\end{figure}\\
Elements in area 1 - 4 are supposed to shift right / down / left / up
respectively. The element in area 5 is the center and does not shift.

When $rows$ is even there is no area 5:\\
\begin{figure}[hbtp]
\begin{verbatim}
1 1 1 2
4 1 2 2
4 4 3 2
4 3 3 3
\end{verbatim}
\end{figure}\\
They can be further divided in the implementation. When $rows$ is odd:\\
\begin{figure}[hbtp]
\begin{verbatim}
  1 1 1 1   2
4   1 1   2 2

4 4   5   2 2

4 4   3 3   2
4   3 3 3 3
\end{verbatim}
\end{figure}\\
When $rows$ is even:\\
\begin{figure}[hbtp]
\begin{verbatim}
  1 1 1   2
4   1   2 2

4 4   3   2
4   3 3 3
\end{verbatim}
\end{figure}\\

To further optimize, it can be observed that the each row of the result matrix
solely depends on up to three rows in the source matrix: the previous row, the
current row, and the next row. A $3 \times rows$ matrix is all the space
required for buffering, and the space complexity is reduced to $O(rows)$. The
output of each result row can be completed as soon as the area 1 of the next
row has been processed. When $rows$ is large, the space is reduced
significantly, as well as the delay between I/O streams.

This algorithm also has the potential of being modified into parallel algorithm
that operates on a matrix. Additional space is required though.

\subsection{Java Code}

The structure of class $RotateArray$:
\begin{lstlisting}[title={<src/rotate/RotateArray.java 29>}, label=Listing29]
package rotate;

import java.util.*;

public class RotateArray {

    <<Input Rotate and Output 31>>

    <<Rotate Main 30>>

}
\end{lstlisting}\begin{footnotesize} \textsc{Included Blocks}: \hyperref[Listing31]{3\-1 on page} \pageref{Listing31}, \hyperref[Listing30]{3\-0 on page} \pageref{Listing30}\end{footnotesize}\vskip 5mm\noindent
The main function:
\begin{lstlisting}[title={<Rotate Main 30>}, label=Listing30]
    public static void main(String[] args) {
        if (!inputRotateAndOutput()) {
            System.out.println("ERROR");
        }
    }
\end{lstlisting}\begin{footnotesize} \textsc{Used in}: \hyperref[Listing29]{s\-r\-c\-/\-r\-o\-t\-a\-t\-e\-/\-R\-o\-t\-a\-t\-e\-A\-r\-r\-a\-y\-.\-j\-a\-v\-a on page} \pageref{Listing29}  \end{footnotesize}\vskip 5mm\noindent
This method is to input, rotate and output.
\begin{lstlisting}[title={<Input Rotate and Output 31>}, label=Listing31]
    /**
      * Reads from input stream, and outputs the result of rotation.
      * Returns true when it is square, and false when not.
      */
    public static boolean inputRotateAndOutput() {
        Scanner sc = new Scanner(System.in);
        int rows = sc.nextInt(); // number of rows
        int halfRows = rows / 2; // half number of rows
        int prev = 2, curr = 0, next = 1; // three row indices for rotation
        int[][] matrix = new int[3][]; // three rows
        String line = sc.nextLine(); // consumes the first link break
        String[] intStrings; // integers in the form of strings
        // initializes matrix
        for (int i = 0; i < 3; i++) {
            matrix[i] = new int[rows];
        }
        // inputs, rotates and outputs 
        for (int i = 0; i < rows; i++) { // row index
            line = sc.nextLine(); // reads in a line
            intStrings = line.split(" "); // split by spaces
            if (intStrings.length == rows) { // if still square
                if (i < halfRows) { // upper half
                    // area 4, shift up
                    for (int j = 0; j < i; j++) { // column index
                        matrix[prev][j] = Integer.parseInt(intStrings[j]);
                    }
                    // outputs the previous row
                    if (i > 0) {
                        for (int j = 0; j < rows; j++) { // column index
                            System.out.print(matrix[prev][j] + (j < rows - 1 ? " " : ""));                
                        }
                        System.out.println();
                    }
                    // area 1, shift right
                    for (int j = i; j < rows - i - 1; j++) { // column index
                        matrix[curr][j + 1] = Integer.parseInt(intStrings[j]);
                    }
                    // area 2, shift down
                    for (int j = rows - i - 1; j < rows; j++) { // column index
                        matrix[next][j] = Integer.parseInt(intStrings[j]);
                    }
                } else if (i > rows - halfRows - 1)  { // lower half
                    // area 4, shift up
                    for (int j = 0; j < rows - i; j++) { // column index
                        matrix[prev][j] = Integer.parseInt(intStrings[j]);
                    }
                    // outputs the previous row
                    if (i > 0) {
                        for (int j = 0; j < rows; j++) { // column index
                            System.out.print(matrix[prev][j] + (j < rows - 1 ? " " : ""));                
                        }
                        System.out.println();
                    }
                    // area 3, shift left
                    for (int j = rows - i; j < i + 1; j++) { // column index
                        matrix[curr][j - 1] = Integer.parseInt(intStrings[j]);
                    }
                    // area 2, shift down
                    for (int j = i + 1; j < rows; j++) { // column index
                        matrix[next][j] = Integer.parseInt(intStrings[j]);
                    }                    
                } else { // mid row
                    for (int j = 0; j < halfRows; j++) {
                        // area 4, shift up
                        matrix[prev][j] = Integer.parseInt(intStrings[j]);
                    }
                    // outputs the previous row
                    if (i > 0) {
                        for (int j = 0; j < rows; j++) { // column index
                            System.out.print(matrix[prev][j] + (j < rows - 1 ? " " : ""));                
                        }
                        System.out.println();
                    }
                    // area 5, no shift
                    matrix[curr][i] = Integer.parseInt(intStrings[i]);
                    for (int j = halfRows + 1; j < rows; j++) {
                        // area 2, shift down
                        matrix[next][j] = Integer.parseInt(intStrings[j]);
                    }
                }
            } else {
                sc.close();
                return false;
            }
            // rotates the rows
            curr = next;
            next = prev;
            prev = 3 - curr - next;
        }
        // outputs the last row
        for (int j = 0; j < rows; j++) { // column index
            System.out.print(matrix[prev][j] + (j < rows - 1 ? " " : ""));                
        }
        System.out.println();
        sc.close();
        return true;
    }
\end{lstlisting}\begin{footnotesize} \textsc{Used in}: \hyperref[Listing29]{s\-r\-c\-/\-r\-o\-t\-a\-t\-e\-/\-R\-o\-t\-a\-t\-e\-A\-r\-r\-a\-y\-.\-j\-a\-v\-a on page} \pageref{Listing29}  \end{footnotesize}\vskip 5mm\noindent
\subsection{Test Cases}

\begin{lstlisting}[title={<rotate0.txt 32>}, label=Listing32]
0
\end{lstlisting}
\begin{lstlisting}[title={<rotate1.txt 33>}, label=Listing33]
1
1
\end{lstlisting}
\begin{lstlisting}[title={<rotate2.txt 34>}, label=Listing34]
2
1 2
3 4
\end{lstlisting}
\begin{lstlisting}[title={<rotate5.txt 35>}, label=Listing35]
5
1 2 3 4 5
6 7 8 9 10
11 12 13 14 15
16 17 18 19 20
21 22 23 24 25
\end{lstlisting}
\begin{lstlisting}[title={<rotate6.txt 36>}, label=Listing36]
6
1 2 3 4 5 6
7 8 9 10 11 12
13 14 15 16 17 18
19 20 21 22 23 24
25 26 27 28 29 30
31 32 33 34 35 36
\end{lstlisting}
\begin{lstlisting}[title={<rotate error.txt 37>}, label=Listing37]
2
1 2
3 4 5
\end{lstlisting}
\end{document}